%LaTeX-inställningar%%%%%%%%%%%%%%%%%%%%%%%%%%%%%%%%%%%%%%%%%%%%%%%%%
% Kompliera med pdflatex
\documentclass[a4paper, 10pt]{article}

\usepackage[utf8x]{inputenc}
\usepackage[swedish]{babel}
\usepackage{graphicx}

\usepackage{geometry} 
\geometry{a4paper} 
\geometry{margin=1in}
\setcounter{section}{1}

\newcommand{\tid}{09:00}
\newcommand{\plats}{Hemifrån}
\newcommand{\datum}{2020-11-09}

\newcommand{\sammankallande}{\textit{Johan Atterfors}}

%Snabbkommandon
\newcommand{\sect}[1][]{\section*{\S \thesection. #1} \stepcounter{section}}
\newcommand{\para}{\paragraph \noindent}
\newcommand{\ssect}[1][]{\subsection*{#1}}

\begin{document}

\section*{\center Kallelse till veckomöte} 
\vspace{1em}
\textbf{Kl:} \tid , \datum \\
\textbf{Plats:} \plats 

\section*{Preliminär dagordning:}
\sect[Mötets öppnande]
\ssect[Närvarande]
Alla
\sect[Informationspunkter]
\begin{itemize}
    \item Planera projektets ramar
    \item Tidsplan
\end{itemize}

\sect[Resultat]
Hållbarhetspoäng:
Baserat på resans alternativ ges poäng på den bästa resan. Resor på landsbygd osv med begränsade alternativ måste också premieras på något sätt. \\
Någon typ av poäng på varje resa i sig. Man samlar inte poäng, det bygger på att informera användaren. Vidare kan även användaren sortera på bättre resor i appen samt välja hur mycket appen ska prioritera resor över älven. \\
Även visa vilka bussar som är elbussar. \\
Information om varje linje ska också finnas på busshållplatsernas tidtabeller. \\
Som tillägg: plantera och väx ett träd. \\
Tydligt markera att man kan åka med båt, inte bara nummret på linjen. Ny färg på linjen i To-go appen? \\
(Långsiktigt förslag; koppla båthållplatser till busshållplatser (framför allt på lindholmen))\\
Information om att ta med cykeln på båten, cykelmarkör.
\\ \ \\
Tidsplan:
Kod klart 18e december. Rapport i mitten av januari. 
\sect[Status]
Ingen har fått nåt gjort än. 
\sect[Kommande uppgifter]
\begin{itemize}
    \item Skriva "några rader om projektet". 
    \subitem Alla samlas kl 14 idag. 
    \item Skapa tydlig plan i typ excel. 
    \subitem Brage, Johan
    \item Klistra ihop en layout av To-go-appen. 
    \subitem Karl, Sebastian
    \item Data att räkna på. Kolla på givna dataset, ställa frågor till VT hur dom räknar. 
    \subitem Alla
\end{itemize}

\sect[Nästa möte sker]
Torsdag 2020-11-12, kl 8 eller kl 9.

\ \\ \
\sammankallande \\

\end{document}